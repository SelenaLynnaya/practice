\documentclass[12pt,a4paper]{scrartcl} 
\usepackage[utf8]{inputenc}
\usepackage[english,russian]{babel}
\usepackage{indentfirst}
\usepackage{misccorr}
\usepackage{graphicx}
\usepackage{indentfirst}
\usepackage{amsmath}
\begin{document}
	\begin{titlepage}
		\begin{center}
			\large
			МИНИСТЕРСТВО НАУКИ И ВЫСШЕГО ОБРАЗОВАНИЯ РОССИЙСКОЙ ФЕДЕРАЦИИ
			
			Федеральное государственное бюджетное образовательное учреждение высшего образования
			
			\textbf{АДЫГЕЙСКИЙ ГОСУДАРСТВЕННЫЙ УНИВЕРСИТЕТ}
			\vspace{0.25cm}
			
			Инженерно-физический факультет
			
			Кафедра автоматизированных систем обработки информации и управления
			\vfill

			\vfill
			
			\textsc{Отчет по практике}\\[5mm]
			
			{\LARGE Программаная реализация \\\LARGE{ Генерация случайных паролей заданной
длины и сложности.}}
			\bigskip
			
			2 курс, группа 2ИВТ 
		\end{center}
		\vfill
		
		\newlength{\ML}
		\settowidth{\ML}{«\underline{\hspace{0.7cm}}» \underline{\hspace{2cm}}}
		\hfill\begin{minipage}{0.5\textwidth}
			Выполнила:\\
			\underline{\hspace{\ML}} А.\,Н.~Лисова\\
			«\underline{\hspace{0.7cm}}» \underline{\hspace{2cm}} 2024 г.
		\end{minipage}%
		\bigskip
		
		\hfill\begin{minipage}{0.5\textwidth}
			Руководитель:\\
			\underline{\hspace{\ML}} С.\,В.~Теплоухов\\
			«\underline{\hspace{0.7cm}}» \underline{\hspace{2cm}} 2024 г.
		\end{minipage}%
		\vfill
		
		\begin{center}
			Майкоп, 2024 г.
		\end{center}
	\end{titlepage}
	
% Содержание
\section{Введение}
\label{sec:intro}

\subsection{Формулировка цели}
Целью данной работы является написание программы для генерации случайных паролей заданной
длины и сложности.

\section{Ход работы}
\subsection{Код выполненной программы}
\begin{verbatim}
#include <iostream>
#include <string>
#include <cstdlib>
#include <ctime>
using namespace std;
// Функция для генерации случайного пароля заданной длины и сложности
string generatePassword(int length, bool includeUppercase, bool includeNumbers) 
{
    // Создаем строку для хранения символов пароля
    string password = "";
    string generatedPassword;
    // Строки символов для различных компонент пароля
    string lowercaseLetters = "abcdefghijklmnopqrstuvwxyz"; 
    string uppercaseLetters = "ABCDEFGHIJKLMNOPQRSTUVWXYZ";
    string numbers = "0123456789"; 
    // Если включены заглавные буквы, добавляем их к списку символов
    if (includeUppercase) 
        password += uppercaseLetters; 
    // Если включены цифры, добавляем их к списку символов
    if (includeNumbers) 
        password += numbers; 
    // Добавляем оставшиеся символы (маленькие буквы) к списку символов
    password += lowercaseLetters;
    // Инициализируем генератор случайных чисел текущим временем
    srand(time(0));
    // Генерируем случайный пароль заданной длины
    for (int i = 0; i < length; ++i) 
    {
        int randomIndex = rand() % password.length();
        generatedPassword += password[randomIndex]; // Добавляем случайный 
        символ к паролю
    }
    return generatedPassword;
}
int main() 
{
    setlocale(0, "");
    // Длина пароля и настройки сложности
    int length, choice;
    bool includeUppercase;
    bool includeNumbers;
    string password;
    // Генерация и вывод пароля
    do 
    {
        cout << "Хотите сгенерировать новый пароль? (1 - да, 0 - нет): ";
        cin >> choice;
        switch (choice) 
        {
        case 1:
            cout << "Введите длину пароля: ";
            cin >> length;
            cout << "Включить заглавные буквы? (1 - да, 0 - нет): ";
            cin >> includeUppercase;
            cout << "Включить цифры? (1 - да, 0 - нет): ";
            cin >> includeNumbers;
            password = generatePassword(length, includeUppercase, includeNumbers);
            cout << "Сгенерированный пароль: " << password << endl;
            break;
        case 0:
            cout << "Выход." << endl;
            break;
        default:
            cout << "Некорректный ввод. Повторите выбор." << endl;
        }
    } while (choice != 0);
    return 0;
}
\end{verbatim}
\begin{figure}[h]
    \centering
	\includegraphics[width=1\textwidth]{possword.jpg}
	\caption{Результат работы}\label{fig:par}
\end{figure}

\end{document}
